% Options for packages loaded elsewhere
\PassOptionsToPackage{unicode}{hyperref}
\PassOptionsToPackage{hyphens}{url}
%
\documentclass[
]{article}
\usepackage{amsmath,amssymb}
\usepackage{lmodern}
\usepackage{iftex}
\ifPDFTeX
  \usepackage[T1]{fontenc}
  \usepackage[utf8]{inputenc}
  \usepackage{textcomp} % provide euro and other symbols
\else % if luatex or xetex
  \usepackage{unicode-math}
  \defaultfontfeatures{Scale=MatchLowercase}
  \defaultfontfeatures[\rmfamily]{Ligatures=TeX,Scale=1}
\fi
% Use upquote if available, for straight quotes in verbatim environments
\IfFileExists{upquote.sty}{\usepackage{upquote}}{}
\IfFileExists{microtype.sty}{% use microtype if available
  \usepackage[]{microtype}
  \UseMicrotypeSet[protrusion]{basicmath} % disable protrusion for tt fonts
}{}
\makeatletter
\@ifundefined{KOMAClassName}{% if non-KOMA class
  \IfFileExists{parskip.sty}{%
    \usepackage{parskip}
  }{% else
    \setlength{\parindent}{0pt}
    \setlength{\parskip}{6pt plus 2pt minus 1pt}}
}{% if KOMA class
  \KOMAoptions{parskip=half}}
\makeatother
\usepackage{xcolor}
\usepackage[margin=1in]{geometry}
\usepackage{color}
\usepackage{fancyvrb}
\newcommand{\VerbBar}{|}
\newcommand{\VERB}{\Verb[commandchars=\\\{\}]}
\DefineVerbatimEnvironment{Highlighting}{Verbatim}{commandchars=\\\{\}}
% Add ',fontsize=\small' for more characters per line
\usepackage{framed}
\definecolor{shadecolor}{RGB}{248,248,248}
\newenvironment{Shaded}{\begin{snugshade}}{\end{snugshade}}
\newcommand{\AlertTok}[1]{\textcolor[rgb]{0.94,0.16,0.16}{#1}}
\newcommand{\AnnotationTok}[1]{\textcolor[rgb]{0.56,0.35,0.01}{\textbf{\textit{#1}}}}
\newcommand{\AttributeTok}[1]{\textcolor[rgb]{0.77,0.63,0.00}{#1}}
\newcommand{\BaseNTok}[1]{\textcolor[rgb]{0.00,0.00,0.81}{#1}}
\newcommand{\BuiltInTok}[1]{#1}
\newcommand{\CharTok}[1]{\textcolor[rgb]{0.31,0.60,0.02}{#1}}
\newcommand{\CommentTok}[1]{\textcolor[rgb]{0.56,0.35,0.01}{\textit{#1}}}
\newcommand{\CommentVarTok}[1]{\textcolor[rgb]{0.56,0.35,0.01}{\textbf{\textit{#1}}}}
\newcommand{\ConstantTok}[1]{\textcolor[rgb]{0.00,0.00,0.00}{#1}}
\newcommand{\ControlFlowTok}[1]{\textcolor[rgb]{0.13,0.29,0.53}{\textbf{#1}}}
\newcommand{\DataTypeTok}[1]{\textcolor[rgb]{0.13,0.29,0.53}{#1}}
\newcommand{\DecValTok}[1]{\textcolor[rgb]{0.00,0.00,0.81}{#1}}
\newcommand{\DocumentationTok}[1]{\textcolor[rgb]{0.56,0.35,0.01}{\textbf{\textit{#1}}}}
\newcommand{\ErrorTok}[1]{\textcolor[rgb]{0.64,0.00,0.00}{\textbf{#1}}}
\newcommand{\ExtensionTok}[1]{#1}
\newcommand{\FloatTok}[1]{\textcolor[rgb]{0.00,0.00,0.81}{#1}}
\newcommand{\FunctionTok}[1]{\textcolor[rgb]{0.00,0.00,0.00}{#1}}
\newcommand{\ImportTok}[1]{#1}
\newcommand{\InformationTok}[1]{\textcolor[rgb]{0.56,0.35,0.01}{\textbf{\textit{#1}}}}
\newcommand{\KeywordTok}[1]{\textcolor[rgb]{0.13,0.29,0.53}{\textbf{#1}}}
\newcommand{\NormalTok}[1]{#1}
\newcommand{\OperatorTok}[1]{\textcolor[rgb]{0.81,0.36,0.00}{\textbf{#1}}}
\newcommand{\OtherTok}[1]{\textcolor[rgb]{0.56,0.35,0.01}{#1}}
\newcommand{\PreprocessorTok}[1]{\textcolor[rgb]{0.56,0.35,0.01}{\textit{#1}}}
\newcommand{\RegionMarkerTok}[1]{#1}
\newcommand{\SpecialCharTok}[1]{\textcolor[rgb]{0.00,0.00,0.00}{#1}}
\newcommand{\SpecialStringTok}[1]{\textcolor[rgb]{0.31,0.60,0.02}{#1}}
\newcommand{\StringTok}[1]{\textcolor[rgb]{0.31,0.60,0.02}{#1}}
\newcommand{\VariableTok}[1]{\textcolor[rgb]{0.00,0.00,0.00}{#1}}
\newcommand{\VerbatimStringTok}[1]{\textcolor[rgb]{0.31,0.60,0.02}{#1}}
\newcommand{\WarningTok}[1]{\textcolor[rgb]{0.56,0.35,0.01}{\textbf{\textit{#1}}}}
\usepackage{graphicx}
\makeatletter
\def\maxwidth{\ifdim\Gin@nat@width>\linewidth\linewidth\else\Gin@nat@width\fi}
\def\maxheight{\ifdim\Gin@nat@height>\textheight\textheight\else\Gin@nat@height\fi}
\makeatother
% Scale images if necessary, so that they will not overflow the page
% margins by default, and it is still possible to overwrite the defaults
% using explicit options in \includegraphics[width, height, ...]{}
\setkeys{Gin}{width=\maxwidth,height=\maxheight,keepaspectratio}
% Set default figure placement to htbp
\makeatletter
\def\fps@figure{htbp}
\makeatother
\setlength{\emergencystretch}{3em} % prevent overfull lines
\providecommand{\tightlist}{%
  \setlength{\itemsep}{0pt}\setlength{\parskip}{0pt}}
\setcounter{secnumdepth}{-\maxdimen} % remove section numbering
\ifLuaTeX
  \usepackage{selnolig}  % disable illegal ligatures
\fi
\IfFileExists{bookmark.sty}{\usepackage{bookmark}}{\usepackage{hyperref}}
\IfFileExists{xurl.sty}{\usepackage{xurl}}{} % add URL line breaks if available
\urlstyle{same} % disable monospaced font for URLs
\hypersetup{
  pdftitle={Homework 2},
  pdfauthor={PSTAT 131/231},
  hidelinks,
  pdfcreator={LaTeX via pandoc}}

\title{Homework 2}
\author{PSTAT 131/231}
\date{}

\begin{document}
\maketitle

{
\setcounter{tocdepth}{2}
\tableofcontents
}
\hypertarget{linear-regression}{%
\subsection{Linear Regression}\label{linear-regression}}

For this lab, we will be working with a data set from the UCI
(University of California, Irvine) Machine Learning repository
(\href{http://archive.ics.uci.edu/ml/datasets/Abalone}{see website
here}). The full data set consists of \(4,177\) observations of abalone
in Tasmania. (Fun fact:
\href{https://en.wikipedia.org/wiki/Tasmania}{Tasmania} supplies about
\(25\%\) of the yearly world abalone harvest.)

\begin{figure}
\centering
\includegraphics[width=1.58333in,height=\textheight]{https://cdn.shopify.com/s/files/1/1198/8002/products/1d89434927bffb6fd1786c19c2d921fb_2000x_652a2391-5a0a-4f10-966c-f759dc08635c_1024x1024.jpg?v=1582320404}
\caption{\emph{Fig 1. Inside of an abalone shell.}}
\end{figure}

The age of an abalone is typically determined by cutting the shell open
and counting the number of rings with a microscope. The purpose of this
data set is to determine whether abalone age (\textbf{number of rings +
1.5}) can be accurately predicted using other, easier-to-obtain
information about the abalone.

The full abalone data set is located in the
\texttt{\textbackslash{}data} subdirectory. Read it into \emph{R} using
\texttt{read\_csv()}. Take a moment to read through the codebook
(\texttt{abalone\_codebook.txt}) and familiarize yourself with the
variable definitions.

Make sure you load the \texttt{tidyverse} and \texttt{tidymodels}!

\begin{Shaded}
\begin{Highlighting}[]
\NormalTok{abalone }\OtherTok{\textless{}{-}}\FunctionTok{read\_csv}\NormalTok{(}\StringTok{"C:/Users/lisha/Downloads/abalone.csv"}\NormalTok{)}
\end{Highlighting}
\end{Shaded}

\textbf{\#\#\# Question 1}

Your goal is to predict abalone age, which is calculated as the number
of rings plus 1.5. Notice there currently is no \texttt{age} variable in
the data set. Add \texttt{age} to the data set.

\begin{Shaded}
\begin{Highlighting}[]
\DocumentationTok{\#\#library(dplyr)}
\NormalTok{abalone }\OtherTok{\textless{}{-}}\NormalTok{abalone }\SpecialCharTok{\%\textgreater{}\%}
  \FunctionTok{mutate}\NormalTok{(}\AttributeTok{age=}\NormalTok{rings}\FloatTok{+1.5}\NormalTok{)}
\NormalTok{abalone}
\end{Highlighting}
\end{Shaded}

\begin{verbatim}
## # A tibble: 4,177 x 10
##    type  longest_sh~1 diame~2 height whole~3 shuck~4 visce~5 shell~6 rings   age
##    <chr>        <dbl>   <dbl>  <dbl>   <dbl>   <dbl>   <dbl>   <dbl> <dbl> <dbl>
##  1 M            0.455   0.365  0.095   0.514  0.224   0.101    0.15     15  16.5
##  2 M            0.35    0.265  0.09    0.226  0.0995  0.0485   0.07      7   8.5
##  3 F            0.53    0.42   0.135   0.677  0.256   0.142    0.21      9  10.5
##  4 M            0.44    0.365  0.125   0.516  0.216   0.114    0.155    10  11.5
##  5 I            0.33    0.255  0.08    0.205  0.0895  0.0395   0.055     7   8.5
##  6 I            0.425   0.3    0.095   0.352  0.141   0.0775   0.12      8   9.5
##  7 F            0.53    0.415  0.15    0.778  0.237   0.142    0.33     20  21.5
##  8 F            0.545   0.425  0.125   0.768  0.294   0.150    0.26     16  17.5
##  9 M            0.475   0.37   0.125   0.509  0.216   0.112    0.165     9  10.5
## 10 F            0.55    0.44   0.15    0.894  0.314   0.151    0.32     19  20.5
## # ... with 4,167 more rows, and abbreviated variable names 1: longest_shell,
## #   2: diameter, 3: whole_weight, 4: shucked_weight, 5: viscera_weight,
## #   6: shell_weight
\end{verbatim}

Assess and describe the distribution of \texttt{age}.

\begin{Shaded}
\begin{Highlighting}[]
\NormalTok{abalone }\SpecialCharTok{\%\textgreater{}\%}
  \FunctionTok{ggplot}\NormalTok{(}\FunctionTok{aes}\NormalTok{(}\AttributeTok{x=}\NormalTok{age))}\SpecialCharTok{+}\FunctionTok{geom\_histogram}\NormalTok{()}
\end{Highlighting}
\end{Shaded}

\includegraphics{homework-2_files/figure-latex/unnamed-chunk-3-1.pdf}

\textbf{Observation: The distribution of age is a little bit right
skewed. Most of the abalone are between 5 and 17 years old.}

\textbf{\#\#\# Question 2}

Split the abalone data into a training set and a testing set. Use
stratified sampling. You should decide on appropriate percentages for
splitting the data.

\emph{Remember that you'll need to set a seed at the beginning of the
document to reproduce your results.}

\begin{Shaded}
\begin{Highlighting}[]
\NormalTok{abalone }\OtherTok{\textless{}{-}}\NormalTok{ abalone }\SpecialCharTok{\%\textgreater{}\%} \FunctionTok{select}\NormalTok{(}\SpecialCharTok{{-}}\NormalTok{rings)}
\FunctionTok{set.seed}\NormalTok{(}\DecValTok{403}\NormalTok{)}
\NormalTok{data\_split }\OtherTok{\textless{}{-}}\FunctionTok{initial\_split}\NormalTok{(abalone,}\AttributeTok{prop=}\FloatTok{0.8}\NormalTok{,}\AttributeTok{strate=}\NormalTok{age) }\CommentTok{\# stratified sampling based on age, 80\% to be the training data}
\NormalTok{data\_train }\OtherTok{\textless{}{-}}\FunctionTok{training}\NormalTok{(data\_split)}
\NormalTok{data\_test }\OtherTok{\textless{}{-}}\FunctionTok{testing}\NormalTok{(data\_split)}
\end{Highlighting}
\end{Shaded}

\textbf{\#\#\# Question 3}

Using the \textbf{training} data, create a recipe predicting the outcome
variable, \texttt{age}, with all other predictor variables. Note that
you should not include \texttt{rings} to predict \texttt{age}. Explain
why you shouldn't use \texttt{rings} to predict \texttt{age}.

Steps for your recipe:

\begin{enumerate}
\def\labelenumi{\arabic{enumi}.}
\item
  dummy code any categorical predictors
\item
  create interactions between

  \begin{itemize}
  \tightlist
  \item
    \texttt{type} and \texttt{shucked\_weight},
  \item
    \texttt{longest\_shell} and \texttt{diameter},
  \item
    \texttt{shucked\_weight} and \texttt{shell\_weight}
  \end{itemize}
\item
  center all predictors, and
\item
  scale all predictors.
\end{enumerate}

You'll need to investigate the \texttt{tidymodels} documentation to find
the appropriate step functions to use.

\begin{Shaded}
\begin{Highlighting}[]
\NormalTok{simple\_data\_recipe }\OtherTok{\textless{}{-}}
  \FunctionTok{recipe}\NormalTok{(age}\SpecialCharTok{\textasciitilde{}}\NormalTok{.,}\AttributeTok{data=}\NormalTok{data\_train) }\SpecialCharTok{\%\textgreater{}\%}
  \FunctionTok{step\_dummy}\NormalTok{(}\FunctionTok{all\_nominal\_predictors}\NormalTok{())}\SpecialCharTok{\%\textgreater{}\%} \CommentTok{\# convert categorical data to numerical data}
  \FunctionTok{step\_interact}\NormalTok{(}\AttributeTok{terms=} \SpecialCharTok{\textasciitilde{}} \FunctionTok{starts\_with}\NormalTok{(}\StringTok{"type"}\NormalTok{)}\SpecialCharTok{:}\NormalTok{shucked\_weight}\SpecialCharTok{+}\NormalTok{longest\_shell}\SpecialCharTok{:}\NormalTok{diameter}\SpecialCharTok{+}\NormalTok{shucked\_weight}\SpecialCharTok{:}\NormalTok{shell\_weight) }\SpecialCharTok{\%\textgreater{}\%} \CommentTok{\# for categorical value, use starts\_with()}
  \FunctionTok{step\_normalize}\NormalTok{(}\FunctionTok{all\_predictors}\NormalTok{()) }\CommentTok{\# help center and scale data}
\end{Highlighting}
\end{Shaded}

\textbf{Explanation: We should not use rings to predict age because
these two variables are essentially the same. There is really no need to
predict age if we already know the rings of abalone. We are, however,
interested in the relationship between rings and other predictors.}

\textbf{\#\#\# Question 4}

Create and store a linear regression object using the \texttt{"lm"}
engine.

\begin{Shaded}
\begin{Highlighting}[]
\NormalTok{lm\_model }\OtherTok{\textless{}{-}} \FunctionTok{linear\_reg}\NormalTok{() }\SpecialCharTok{\%\textgreater{}\%} 
  \FunctionTok{set\_engine}\NormalTok{(}\StringTok{"lm"}\NormalTok{)}
\end{Highlighting}
\end{Shaded}

\textbf{\#\#\# Question 5}

Now:

\begin{enumerate}
\def\labelenumi{\arabic{enumi}.}
\tightlist
\item
  set up an empty workflow,
\item
  add the model you created in Question 4, and
\item
  add the recipe that you created in Question 3.
\end{enumerate}

\begin{Shaded}
\begin{Highlighting}[]
\NormalTok{lm\_wflow }\OtherTok{\textless{}{-}} \FunctionTok{workflow}\NormalTok{() }\SpecialCharTok{\%\textgreater{}\%} 
  \FunctionTok{add\_model}\NormalTok{(lm\_model) }\SpecialCharTok{\%\textgreater{}\%} 
  \FunctionTok{add\_recipe}\NormalTok{(simple\_data\_recipe)}
\end{Highlighting}
\end{Shaded}

\textbf{\#\#\# Question 6}

Use your \texttt{fit()} object to predict the age of a hypothetical
female abalone with longest\_shell = 0.50, diameter = 0.10, height =
0.30, whole\_weight = 4, shucked\_weight = 1, viscera\_weight = 2,
shell\_weight = 1.

\begin{Shaded}
\begin{Highlighting}[]
\NormalTok{lm\_fit }\OtherTok{\textless{}{-}} \FunctionTok{fit}\NormalTok{(lm\_wflow, data\_train)}
\NormalTok{new\_data }\OtherTok{\textless{}{-}} \FunctionTok{tibble}\NormalTok{(}\AttributeTok{longest\_shell =} \FloatTok{0.50}\NormalTok{, }\AttributeTok{diameter =} \FloatTok{0.10}\NormalTok{, }\AttributeTok{height =} \FloatTok{0.30}\NormalTok{, }\AttributeTok{whole\_weight =} \DecValTok{4}\NormalTok{, }\AttributeTok{shucked\_weight =} \DecValTok{1}\NormalTok{, }\AttributeTok{viscera\_weight =} \DecValTok{2}\NormalTok{, }\AttributeTok{shell\_weight =} \DecValTok{1}\NormalTok{,}\AttributeTok{type=}\StringTok{"F"}\NormalTok{)}
\NormalTok{data\_train\_result }\OtherTok{\textless{}{-}} \FunctionTok{predict}\NormalTok{(lm\_fit,new\_data)}
\NormalTok{data\_train\_result }
\end{Highlighting}
\end{Shaded}

\begin{verbatim}
## # A tibble: 1 x 1
##   .pred
##   <dbl>
## 1  23.9
\end{verbatim}

\textbf{\#\#\# Question 7}

Now you want to assess your model's performance. To do this, use the
\texttt{yardstick} package:

\begin{enumerate}
\def\labelenumi{\arabic{enumi}.}
\tightlist
\item
  Create a metric set that includes \emph{R\textsuperscript{2}}, RMSE
  (root mean squared error), and MAE (mean absolute error).
\item
  Use \texttt{predict()} and \texttt{bind\_cols()} to create a tibble of
  your model's predicted values from the \textbf{training data} along
  with the actual observed ages (these are needed to assess your model's
  performance).
\item
  Finally, apply your metric set to the tibble, report the results, and
  interpret the \emph{R\textsuperscript{2}} value.
\end{enumerate}

\begin{Shaded}
\begin{Highlighting}[]
\FunctionTok{library}\NormalTok{(yardstick)}
\NormalTok{data\_metrics }\OtherTok{\textless{}{-}} \FunctionTok{metric\_set}\NormalTok{(rmse,rsq,mae)}
\NormalTok{data\_train\_all\_result }\OtherTok{\textless{}{-}} \FunctionTok{predict}\NormalTok{ (lm\_fit,}\AttributeTok{new\_data=}\NormalTok{data\_train}\SpecialCharTok{\%\textgreater{}\%}\FunctionTok{select}\NormalTok{(}\SpecialCharTok{{-}}\NormalTok{age))}
\NormalTok{final\_predict }\OtherTok{\textless{}{-}}\FunctionTok{bind\_cols}\NormalTok{(data\_train\_all\_result,data\_train }\SpecialCharTok{\%\textgreater{}\%} \FunctionTok{select}\NormalTok{(age)) }
\NormalTok{final\_predict}
\end{Highlighting}
\end{Shaded}

\begin{verbatim}
## # A tibble: 3,341 x 2
##    .pred   age
##    <dbl> <dbl>
##  1 14.7   12.5
##  2 12.2   10.5
##  3 13.4   11.5
##  4 10.1    8.5
##  5 11.5   14.5
##  6 11.6   10.5
##  7  8.81  12.5
##  8 12.3   11.5
##  9 11.7   12.5
## 10 11.9   12.5
## # ... with 3,331 more rows
\end{verbatim}

\begin{Shaded}
\begin{Highlighting}[]
\FunctionTok{data\_metrics}\NormalTok{(final\_predict,}\AttributeTok{truth=}\NormalTok{age,}\AttributeTok{estimate =}\NormalTok{ .pred)}
\end{Highlighting}
\end{Shaded}

\begin{verbatim}
## # A tibble: 3 x 3
##   .metric .estimator .estimate
##   <chr>   <chr>          <dbl>
## 1 rmse    standard       2.14 
## 2 rsq     standard       0.557
## 3 mae     standard       1.54
\end{verbatim}

\textbf{The root mean square error is 2.138, the coefficient of
determination is 0.5567 and the mean absolute error is 1.5438. In terms
of coefficient of determination, 0.5567 means that 55.67\% of
variability observed in the target variable is explained by the
regression model. Since this value is not very large compared to 1, this
indicates that the regression model does not perform very well in
predicting the target variable.}

\hypertarget{required-for-231-students}{%
\subsubsection{Required for 231
Students}\label{required-for-231-students}}

In lecture, we presented the general bias-variance tradeoff, which takes
the form:

\[
E[(y_0 - \hat{f}(x_0))^2]=Var(\hat{f}(x_0))+[Bias(\hat{f}(x_0))]^2+Var(\epsilon)
\]

where the underlying model \(Y=f(X)+\epsilon\) satisfies the following:

\begin{itemize}
\tightlist
\item
  \(\epsilon\) is a zero-mean random noise term and \(X\) is non-random
  (all randomness in \(Y\) comes from \(\epsilon\));
\item
  \((x_0, y_0)\) represents a test observation, independent of the
  training set, drawn from the same model;
\item
  \(\hat{f}(.)\) is the estimate of \(f\) obtained from the training
  set.
\end{itemize}

\textbf{\#\#\#\# Question 8}

Which term(s) in the bias-variance tradeoff above represent the
reproducible error? Which term(s) represent the irreducible error?

\textbf{\(Var(\hat{f}(x_0))\) and \([Bias(\hat{f}(x_0))]^2\) are the
reducible errors, whereas \(Var(\epsilon)\) is the irreducible error.}

\textbf{\#\#\#\# Question 9}

Using the bias-variance tradeoff above, demonstrate that the expected
test error is always at least as large as the irreducible error.

\textbf{The reason why expected test error is always at least as large
as the irreducible error is because even if we minimize the reducible
errors to 0, i.e, when we take} \(\hat{f}(x_0)=E[Y|X=x_0]\)\textbf{, we
still leave with irreducible error \(Var(\epsilon)\) in the
bias-variance tradeoff, meaning that expect test error should always be
greater than or equal to irreducible error.}

\textbf{\#\#\#\# Question 10}

Prove the bias-variance tradeoff.

Hints:

\begin{itemize}
\tightlist
\item
  use the definition of \(Bias(\hat{f}(x_0))=E[\hat{f}(x_0)]-f(x_0)\);
\item
  reorganize terms in the expected test error by adding and subtracting
  \(E[\hat{f}(x_0)]\)
\end{itemize}

\textbf{Please see the picture that I uploaded on Gauchospace in the
same file as HW2. }

\end{document}
